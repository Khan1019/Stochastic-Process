\documentclass{article}
\usepackage{ctex}
\usepackage{geometry}
\usepackage{amsmath}
\usepackage{amssymb}
\usepackage{multirow}
\usepackage{graphicx}
\usepackage{siunitx}
\usepackage{float}
\usepackage{hyperref}
\title{随机过程B\ 2020-2021学年第二学期试卷解答}
\author{陈镜舟}
\begin{document}
    \maketitle
    \setlength{\parindent}{0pt}
    \section*{一}
    \subsection*{1}
    (1)错,应该是独立增量\\
    (2)对,比如$P=\begin{pmatrix}1&0\\0&1\end{pmatrix}$\\
    (3)错。反例:一维对称随机游走是零常返。\\
    (4)错,反例即(2)中的$P$。\\
    (5)对。周期无穷大$\Rightarrow P_{ii}^{(n)}=0$。
    \subsection*{2}
    计算积分时使用了$\Gamma$积分公式。
    \begin{align*}
        &\int_{0}^{+\infty}\mu e^{-\mu t} \frac{(\lambda t)^{k}e^{-\lambda t}}{k!}dt\\
        =&\frac{u\lambda^{k}}{k!}\int_{0}^{+\infty}t^{k}e^{-(\lambda+u)t}dt\\
        =&\frac{\mu\lambda^{k}}{k!}\cdot\frac{1}{(\lambda+\mu)^{k+1}}\int_{0}^{+\infty}(\lambda+\mu)^{k}t^{k}e^{-(\lambda+\mu)t}d(\lambda+\mu)t\\
        =&\frac{\mu\lambda^{k}}{k!}\cdot\frac{1}{(\lambda+\mu)^{k+1}}\cdot\int_{0}^{+\infty}u^{k}e^{-u}du\\
        =&\frac{u\lambda^{k}}{k!}\frac{1}{(\lambda+\mu)^{k+1}}\cdot k!\\
        =&\frac{\mu\lambda^{k}}{(\lambda+\mu)^{k+1}}
    \end{align*}
    \subsection*{3}
    C,因为$E(N(t))=\lambda t\neq const$。\\
    \subsection*{4}
    $N(t)$即参数为$\mu$的Poisson过程。
    \begin{align*}
        &f_{s_{n}}(x)=\mu e^{-\mu x}\cdot\frac{(\mu x)^{n-1}}{(n-1)!}\\
        &P(N(t)=n)=\frac{(\mu t)^{n}e^{-\mu t}}{n!}
    \end{align*}
    \subsection*{5}
    D,其他选项的错误原因如下。 \\
    A:N和M不一定独立。\\
    B:Poisson过程的时间间隔服从指数分布,因此每隔一辆车记录一次得到的随机过程的时间间隔服从$\Gamma$分布,这显然不是Poisson过程。\\
    C:$R$应在$\tau=0$处取得最大值。
    \subsection*{6}
    思路:化为标准高斯分布。\\
    $R(\tau)=\frac{1}{2}e^{-|\tau|}\Rightarrow R(0)=\frac{1}{2}\Rightarrow \sigma=\frac{1}{\sqrt{2}}$。
    因此将区间化为$[\frac{0.5-0}{1/\sqrt{2}},\frac{1-0}{1/\sqrt{2}}]$,即$[\frac{1}{\sqrt{2}},\sqrt{2}]$。故
    \begin{align*}
        P=\Phi(\sqrt{2})-\Phi(\frac{1}{\sqrt{2}})
    \end{align*}
    \subsection*{7}
    $\lambda re^{-\lambda r}$,等价于参数为$\lambda$的Poisson过程在$[0,r]$时间段内观察到一次的概率。
    \section*{二}

复合Poisson过程,参考教材第9页例1.12和第21页。\\
\begin{align*}
    &X_{(t)}=\sum_{i=1}^{N(t)}Y_{i}\\
    &Y_{i}\sim\begin{pmatrix}30&50\\0.75&0.25\end{pmatrix},\ EY=35,\ VarY=75\\
    &E[X(t)]=10\times35t=350t,\ Var[X(t)]=10\times(75+35^{2})t=13000t\\
\end{align*}
\begin{align*}    
    g_{Y}(\mu)&=\frac{3}{4}e^{30\mu}+\frac{1}{4}e^{50\mu}\\
    g_{X(t)}(u)&=E[(g_{Y}(u))^{N}]=\sum_{k=0}^{+\infty}\frac{(\lambda t)^{k}\cdot e^{-\lambda t}}{k!}\cdot g_{Y}^{k}=\frac{1}{e^{\lambda t}}\sum_{k=0}^{+\infty}\frac{(g_{Y}\lambda t)^{k}}{k!}\\
    &=e^{\lambda t(g_{Y}-1)}=exp\{10t(\frac{3}{4}e^{30u}+\frac{1}{4}e^{50u}-1)\}
\end{align*}
\section*{三}
\subsection*{1}
不可约,非周期,正常反,矩阵略。
\subsection*{2}
各个状态都是遍历态,计算可得
\begin{align*}
    \pi_{i}=\frac{\frac{1}{1-P_{ii}}}{\sum_{j=1}^{a}\frac{1}{1-P_{jj}}}
\end{align*}
\section*{四}
\subsection*{1}
显然。
\subsection*{2}
各状态互达且状态1为非周期,故各个态均为非周期。
\begin{align*}
    f_{11}&=(1-p_{1})+p_{1}(1-p_{2})+p_{1}p_{2}(1-p_{3})+\cdots\\
    &=1-p_{1}+p_{1}-p_{1}p_{2}+p_{1}p_{2}-p_{1}p_{2}p_{3}+\cdots=1
\end{align*}
状态1常返$\Rightarrow$各个态均常返。
\section*{五}
\subsection*{1}
\begin{align*}
    E[Y(t)]&=E[X(t)]\cdot\int_{0}^{2\pi}\frac{1}{2\pi}\cos(\omega_{0}t+\theta)d\theta=0\\
    R_{Y}(\tau)&=E[X(t)\cos(w_{0}t+\theta)X(s)\cos(w_{0}s+\theta)]\\
    &=E[X(t)X(s)]\cdot E[\frac{\cos(w_{0}t+w_{0}s+2\theta)+\cos(w_{0}t-w_{0}s)}{2}]\\
    &=R_{X}(\tau)\cdot\frac{1}{2}\cos w_{0}\tau \\
    E[Y^{2}(t)]&=R_{Y}(0)=\frac{1}{2}R_{X}(0)<\infty 
\end{align*}
所以$Y(t)$为平稳过程。
\subsection*{2}
\begin{align*}
    S_{Y}(w)&=\int R_{Y}(x)e^{-jw\tau}d\tau \\
    &=\int\frac12R_{X}(\tau)\cdot\frac12[e^{jw_{0}\tau}+e^{-jw_{0}\tau}]\cdot e^{-jw\tau}d\tau\\
    &= \frac{1}{4}\int R_{x}(\tau)[e^{-j(w-w_{0})\tau}+e^{-j(w+w_{0})\tau}]d\tau \\
    &=\frac{1}{4}[S_{x}(w-w_{0})+S_{x}(w+w_{0})]
\end{align*}
\section*{六}
\subsection*{1}
\begin{align*}
    R(\tau)=\frac{1}{2\pi}\int\frac{w^{2}+2}{w^{4}+7w^{2}+12}e^{jw\tau}dw=\frac{1}{2}e^{-2|\tau|}-\frac{\sqrt{3}}{6}e^{-\sqrt{3}|\tau|}
\end{align*}
\subsection*{2}
\begin{align*}
    \int_{-\infty}^{+\infty}|R(\tau)|d\tau \leq\int_{0}^{+\infty}e^{-2t}+\frac{\sqrt{3}}{3}e^{-\sqrt{3}t}dt<\infty 
\end{align*}
因此有均值遍历性。
\end{document}